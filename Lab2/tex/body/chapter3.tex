%
%
%\section{预处理}
%\subsection{预处理过程概述}
%预处理阶段会处理预编译指令,包括绝大多数的 \# 开头的指令,如 include define if 等等,对
%include 指令会替换对应的头文件,对 define 的宏命令会直接替换相应内容,同时会删除注释,添
%加行号和文件名标识。\\
%对于 gcc,通过添加参数-E 令 gcc 只进行预处理过程,参数-o 改变 gcc 输出文件名,因此通过命
%令 gcc main.c -E -o main.i,即可得到预处理后文件。\\
%观察预处理文件,可以发现文件长度远大于源文件,这就是将代码中的头文件进行了替代导致的
%结果。另外,实际上预处理过程是 gcc 调用了另一个程序(C Pre-Processor 调用时简写作 cpp)完成
%的过程,有兴趣的同学可以自行尝试。\\
%
%%\subsection{预处理的目的}
%%预处理的功能主要是对源程序进行一些文本层面的处理。之所以要设置预处理环节,是因为预处理能够方便代码的编写,也能提高程序的运行效率。\\
%%
%%以程序中的宏定义为例。一方面,通过在程序中宏定义需要的常量,我们在编程时对于程序的修改将方便很多。另一方面,在函数调用时,使用带参数的宏定义完成参数传递,可以减少系统开销,提高运行效率。这是因为如果在预处理阶段即进行了宏展开,那么程序在执行时就不需要再去转换,在当地执行即可取出参数的值。\\
%
%\subsection{预处理语句}
%\subsubsection{预处理语句的功能}
%除了预先定义的宏之外,预处理器所有的功能均由预处理语句触发。下面依照GCC官方文档,对源程序中预处理语句的功能予以罗列:
%
%\begin{itemize}
%	\item \textbf{包含头文件}(Inclusion of header files):这些是可以替换到您的程序中的声明文件。
%	
%	\item \textbf{宏展开}(Macro expansion):您可以定义宏,它们是C代码
%	的任意片段的缩写。预处理器将在整个程
%	序中用它们的定义代替宏。某些宏会自动
%	为您定义。
%	
%	\item \textbf{条件编译}(Conditional compilation):您可以根据各种条件引入或去除部分程序。
%	
%	\item \textbf{行控制}(Line control):如果您想使用您的程序把一些源文件组合起来或者重新编排,将其生成一个中间文件,之后再去编译这个中间文件的话,您可以使用行控制来告诉编译器该中间文件的每一行来自哪里。	
%	
%	\item \textbf{诊断}(Diagnostics):您可以在编译时检测问题,并发出错误或警告。		
%	
%\end{itemize}
%
%\subsubsection{常用预处理命令}
%预处理命令由\#开头,它独占一行,\#之前只能是空白符。在C/C++语言源文件中,以\#开头的语句就是预处理语句,不以\#开头的语句为C/C++中的代码行。常用的预处理命令列举如下:
%\begin{table}[H]
%	\caption{常用预处理命令简介}
%	\centering
%	\begin{tabular}{l|l}
%		\toprule
%		预处理命令 &  \quad \quad \quad \quad \quad \quad \quad \quad \quad \quad 对应功能 \\
%		\midrule[1.5pt]
%		\#define & 定义一个预处理宏   \\
%		\#undef & 取消宏的定义  \\
%		\#include & 包含一个文件   \\
%		\#if & 预处理语法中的条件命令,
%		相当于C语法中的 if 语句   \\
%		
%		\#ifdef & 判断某个宏是否被定义,
%		若已定义,执行随后的语句   \\
%		
%		\#ifndef & 判断某个宏是否未被定义,
%		若未定义,执行随后的语句   \\
%		
%		\# else & 与\#if,\#ifdef,\#ifndef 对应,若这些条件不满足,\\
%		
%		 & 则执行\#else 之后的语句,相当于C语法中的else\\
%
%		\# elif & 若\#if,\#ifdef,\#ifndef 或之前的\#elif条件不满足,\\
%		
%		 & 则执行\#elif 之后的语句,相当于C语法中的 else-if\\		 
%		 
%		\#endif & \#if,\#ifdef,\#ifndef 这些条件命令的结束标志   \\
%		
%		
%		\#pragma & 说明编译器信息   \\
%		\#warning & 显示编译警告信息   \\
%		\#error & 显示编译错误信息   \\
%		
%		\bottomrule
%	\end{tabular}
%\end{table}
%
%%\subsubsection{预处理语句的文法}
%%预处理并不分析整个源代码文件,它只是将源代码分割成一些标记(token),识别语句中哪些是 C语句,哪些是预处理语句。\\
%%
%%预处理语句的一般格式如下:
%%
%%\begin{verbatim}
%%      #command name(...) token(s) 
%%\end{verbatim}
%%
%%其中的参数含义如下:
%%\begin{verbatim}
%%      command:预处理命令的名称,它之前以#开头,#之后紧随预处理命令,通常不能有空格。 若某行中只包含#以及空白符,那么在C语言中该行被理解为空白。
%%      name:代表宏名称,它可带参数,参数可以是可变参数列表。
%%      \:语句中可以利用"\"来换行。
%%\end{verbatim}
%
%\subsection{例程输出分析}
%
%\subsubsection{预处理器调用命令}
%Linux中使用gcc驱动器调用预处理器,处理源文件test.c得到预处理输出文件test.i的命令是:
%
%\begin{verbatim}
%      gcc test.c -E -o test.i
%\end{verbatim}
%
%其中的-E参数可以让GCC只对源代码进行预处理而不进行后续编译操作,并将与处理之后的代码输出到命令中指定的输出文件中。
%
%\subsubsection{例程输出结果}
%
%test.c经过预处理之后得到test.i,执行过程及部分文件内容如下所示:
%
%%\begin{figure}[H]
%%	\centering
%%	\includegraphics[width=16 CM]{figure/0062}
%%	\caption{预处理源程序test.c}
%%	\label{fig:预处理源程序test.c}
%%\end{figure}
%
%%\begin{figure}[H]
%%	\centering
%%	\includegraphics[width=16 CM]{figure/0062}
%%	\caption{test.c文件内容}
%%	\label{fig:test.c文件内容}
%%\end{figure}
%
%%\begin{figure}[H]
%%	\centering
%%	\includegraphics[width=16 CM]{figure/0082}
%%	\caption{test.i部分文件内容-1}
%%	\label{fig:test.i部分文件内容-1}
%%\end{figure}
%
%%\begin{figure}[H]
%%	\centering
%%	\includegraphics[width=16 CM]{figure/0092}
%%	\caption{test.i部分文件内容-2}
%%	\label{fig:test.i部分文件内容-2}
%%\end{figure}
%
%
%%此外,还对源程序进行局部修改之后查看了输出结果,相关操作将在下一部分【 例程结果分析】中详细说明。
%
%%\subsubsection{例程输出结果分析}
%%通过上述实验及系列其他实验( 下文会详细说明),我们可以发现,预处理器对于源文件进行了一些处理,替换了一些内容,丢弃了一些内容,保留了一些内容,增加了一些内容:
%%
%%\begin{enumerate}
%%	\item \textbf{替换部分内容}:我们可以发现生成的factorial.i的文件长度远长于文件factorial.cpp的文件长度。这是因为预处理器会处理以\#开头的命令,根据命令的内容进行宏替换、处理条件预处理指令、进行文件包含替换等。例如本实验中源程序的"\#include<iostream>"语句就没有出现在输出结果中,相应的C++头文件被替换进了源文件中。
%%	
%%	\begin{figure}[H]
%%		\centering
%%		\includegraphics[height=6 CM, width=16 CM]{figure/010}
%%		\caption{factorial.i部分文件内容-3}
%%		\label{fig:factorial.i部分文件内容-3}
%%	\end{figure}
%%
%%    除了factorial.cpp涉及的文件包含预处理指令的替换,预处理器还会进行如下替换相关操作:
%%
%%\begin{itemize}
%%	\item \textbf{宏定义替换}:所有宏定义行会被空白行替代,所有使用宏定义的位置会被实际内容替换。\\
%%	
%%	例如在如下实验中,test1-1.cpp中pi和abc()两处宏定义就被预处理器识别并展开,而原始的两处宏定义行被替换成了空白行:
%%	
%%	\begin{figure}[H]
%%	\centering
%%	\includegraphics[height=6 CM]{figure/011}
%%	\caption{test1-1.cpp预处理过程中的宏替换}
%%	\label{test1-1.cpp预处理过程中的宏替换}
%%    \end{figure}	
%%	
%%	\item \textbf{注释替换}:所有注释都将被空格或者空行替代。\\
%%	
%%	例如在如下实验中,test1-2.cpp中两处注释就被替换成了空格或者空行:	
%%	\begin{figure}[H]
%%	\centering
%%	\includegraphics[height=7 CM]{figure/012}
%%	\caption{test1-2.cpp预处理过程中的注释替换}
%%	\label{test1-2.cpp预处理过程中的注释替换}
%%    \end{figure}
%%
%%\end{itemize}
%%
%%	\item \textbf{丢弃空行空格}:空白的长行将被丢弃。根据ISO标准规定,在GNU CPP中,空白的行会被丢弃,参数之间的空格被折叠成为单个空格。
%%
%%	\item \textbf{保留布局控制指令}:所有\#pragma布局控制预处理语句会被保留,因为编译器会使用到它们。
%%	
%%	\item \textbf{增加行标}: 
%%	
%%	观察factorial.i可以发现其中多出了许多形如以下的代码:
%%	
%%	\begin{figure}[H]
%%	\centering
%%	\includegraphics[height=7 CM]{figure/013}
%%	\caption{factorial.i中的行标}
%%	\label{factorial.i中的行标}
%%    \end{figure}	
%%	
%%	这些被称为linemarkers(行标),其格式遵循:
%%	
%%	\begin{verbatim}
%%	      # linenum filename flags 
%%	\end{verbatim}
%%	
%%	其中:
%%	\begin{verbatim}
%%	语句中的linenum是指filename中相应文件的对应行。 若某行中只包含#以及空
%%	白符,那么在C语言中该行被理解为空白。
%%	
%%	flags含义如下:
%%	     1:表示引入了一个新文件
%%	     2:表示返回了一个文件(该文件是包含了其他文件之后的文件)
%%	     3:表示以下文本来自系统头文件,因此应该抑制某些警告
%%	     4:表示以下文本应被视为包裹在隐式extern "C"块中    
%%	
%%	\end{verbatim}
%%	
%%	\item \textbf{处理条件编译}:预处理器将根据\#if,\#ifdef,\#ifndef,\#endif 来确定是否需要对各部分进行相应处理。使用条件编译可以使目标程序变小,运行时间变短。\\
%%	
%%	程序test1-3.cpp预处理结果如下:
%%	
%%	\begin{figure}[H]
%%	\centering
%%	\includegraphics[height=7 CM]{figure/014}
%%	\caption{test1-3.cpp预处理过程中的条件编译处理}
%%	\label{test1-3.cpp预处理过程中的条件编译处理}
%%    \end{figure}	
%%
%%
%%
%%	
%%\end{enumerate}
%%
%
%
%
