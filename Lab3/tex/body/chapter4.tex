%
%
%\section{编\quad 译}
%\subsection{编译过程概述}
%编译(compiling)就是通过词法分析和语法分析,在确认所有指令都符合语法规则之后,将其翻译成等价的中间代码或者是汇编代码的过程。\\
%
%编译器是GCC的核心部件,其功能的实现过程也相对复杂。大体来说,GCC编译器首先检查代码是否有语法错误,确认代码无误后,GCC才会继续进行使用预处理器的输出文件生成汇编源文件。\\
%
%
%
%%\subsection{AT\&T格式汇编语言}
%%GCC编译器的输出结果是汇编代码。既然想要对编译过程进行分析,就要了解GCC编译器输出的汇编语言语法。\\
%%
%%Linux内核代码大量使用内嵌汇编,以进行某些特定功能的实现,或对某功能进行快速实现。Linux 系统中使用的汇编语言格式为AT\&T;而我们在以前只接触过Intel格式汇编语言和MIPS格式汇编语言。尽管宏观理念是大体类似的,但是细节处的语法差异对于我们理解程序也有着较大的困阻。\\
%%
%%因此,本部分对AT\&T风格汇编语言和Intel风格汇编语言的主要语法差异进行归纳总结,以期能通过比照学习,尽快熟悉AT\&T汇编语言。
%%
%%\begin{enumerate}
%%	\item \textbf {字母大小写}:\\
%%	Intel格式的指令使用大写字母,而AT\&T格式的使用小写字母。\\
%%	
%%	\item \textbf {操作数赋值方向}:\\
%%	在Intel语法中,第一个表示目的操作数,第二个表示源操作数,赋值方向从右向左;
%%	而在AT\&T语法中,第一个为源操作数,第二个为目的操作数,方向从左到右,合乎自然。\\
%%	\textbf {示例:将ebx的值赋给eax}:\\
%%	Intel:MOV EAX,EBX \\
%%	AT\&T:movl \%ebx,\%eax\\
%%	
%%	\item \textbf {前缀}:\\
%%	在Intel语法中寄存器和立即数不需要前缀;\\在AT\&T 中寄存器需要加前缀“\%” ,而立即数需要加前缀“\$” 。\\
%%	\textbf {示例:将1赋值给eax}:\\
%%	Intel:MOV EAX,1 \\
%%	AT\&T:movl \$1,\%eax \\
%%	
%%	
%%	\textbf {示例:子过程调用}:\\
%%	Intel:CALL FAR SECTION:OFFSET \\
%%	AT\&T:lcall \$secion:\$offset  \\	
%%	
%%	\textbf {示例:远程跳转}:\\
%%	Intel:JMP FAR SECTION:OFFSET \\
%%	AT\&T:ljmp \$secion:\$offset  \\	
%%	
%%	\textbf {示例:调用/跳转返回}:\\
%%	Intel:RET FAR SATCK\_ADJUST \\
%%	AT\&T:lret \$stack\_adjust   \\	
%%	
%%	\item \textbf {间接寻址}:\\
%%	Intel中基地址使用“[” 、“]”;而在 AT\&T 中使用“(”、“)” 。\\
%%	另外两者处理复杂操作数的语法也不同,Intel为“Segreg:[base+index*scale+disp]”;而在AT\&T中为“\%segreg:disp(base,index,sale)”,其中segreg,index,scale,disp都是可选的,在指定index而没有显式指定Scale的情况下使用默认值。\\
%%	\textbf {示例:间接寻址}:\\
%%	Intel:INSTR FOO,SEGREG:[BASE+INDEX*SCALE+DISP] \\
%%	AT\&T:instr  \%segreg:disp(base,index,scale),foo 
%%	\\
%%	
%%	\item \textbf {后缀}:\\
%%	AT\&T语法中大部分指令操作码的最后一个字母表示操作数大小, “b”表示 byte(一个字节),“w”表示 word(2 个字节),“l”表示 long(4 个字节);Intel 中处理内存操作数时也有类似的语法,如:BYTE PTR、WORD PTR、DWORD PTR。\\
%%	\textbf {示例:赋值}:\\
%%	Intel:MOV AL, BL \\
%%	AT\&T:movb \%bl,\%al \\
%%
%%	Intel:MOV AX,BX \\
%%	AT\&T:movw \%bx,\%ax \\
%%	
%%	Intel:EAX, DWORD PTR[EBX] \\
%%	AT\&T:movl (\%ebx), \%eax \\
%%	
%%	此外,AT\&T 汇编指令中跳转指令标号后的后缀表示跳转方向, “f” 表示向前 (forward) , “b” 表示向后 (back) 。
%%	\textbf {示例:跳转}:\\
%%	 \quad AT\&T:\\
%%	 \quad XXXXXXX \\
%%	 \quad jmp 1f \\
%%     1: mov \$0x8000C580, \%eax \\
%%
%%\end{enumerate}
%
%\subsection{例程输出分析}
%
%\subsubsection{编译器调用命令}
%Linux中,能让GCC编译器使用预处理器的输出文件test.i生成汇编源文件test.s的命令是:
%
%\begin{verbatim}
%      gcc test.i -S -o test.s
%\end{verbatim}
%
%其中的-S 参数表示 GCC 只生成汇编源代码而不
%进行汇编。\\
%输出的.s 文件——汇编语言源程序,在后
%期阶段不再进行预处理操作,而直接进行汇编操作。\\
%输出的.S 文件——汇编语言源程序,在后
%期阶段还会进行预处理、汇编等操作。
%
%\subsubsection{例程输出结果}
%
%test.i经过编译之后得到test.s,执行过程及部分文件内容如下所示:
%
%%\begin{figure}[H]
%%	\centering
%%	\includegraphics[width=16 CM]{figure/015}
%%	\caption{编译test.i生成test.s}
%%	\label{fig:编译test.i生成test.s}
%%\end{figure}
%
%%\begin{figure}[H]
%%	\centering
%%	\includegraphics[width=16 CM]{figure/0152}
%%	\caption{test.s文件部分内容}
%%	\label{fig:test.s文件部分内容}
%%\end{figure}
%
%%\subsection{编译优化处理}
%%
%%\subsubsection{编译优化概述}
%%编译阶段会对代码进行优化处理,优化处理是编译系统中一项比较艰深的技术。它涉及到的问题不仅同编译技术本身有关,而且同机器的硬件环境也有很大的关系。优化分为两部分:优化一部分是对中间代码的优化,这种优化不依赖于具体的计算机。另一种优化则主要针对目标代码的生成而进行的。
%%
%%\begin{itemize}
%%	\item \textbf{不依赖于计算机硬件结构的优化}:主要是删除 公共表达式、循环优化(  代码外提、强度 削弱、变换循环控制、已知量的合并等)、 无用赋值的删除等。 
%%	
%%	\item \textbf{同计算机硬件结构相关的优化}:主要考虑如 何充分利用机器的硬件寄存器存放的有 关变量的值以减少内存的访问次数;根据 机器硬件执行指令的特点对指令进行调 整使目标代码比较短,执行效率更高等。	
%%\end{itemize}
%%
%%\subsubsection{-O优化选项分析}
%%
%%	 -O 选项可以使编译器对代码进行自动优化编译,输出效率更高的可执行文 件。-O 后面还可以跟上数字指定优化级别,如:-O0、-O1、-O2、-O3 等,其中-O0 这个等级关闭所有的优化选项。没有数字默认是 1,最大可以加到 3,优化级别越高,产生的代码的执行效率就越高,但是编译的过程花费的时间就会越长。
%%	 
%%\begin{itemize}
%%	\item \textbf{关闭优化:-O0}:\\
%%	设置为这个优化等级将关闭所有的优化选项。\\
%%	对应命令为:
%%	\begin{verbatim}
%%	      g++ -O0 -S factorial.i -o factorial_O0.s
%%	\end{verbatim}
%%	
%%\begin{figure}[H]
%%	\centering
%%	\includegraphics[width=16 CM]{figure/017}
%%	\caption{O0等级优化:factorial\_O0.s部分内容}
%%	\label{fig:O0等级优化:factorial_O0.s部分内容}
%%\end{figure}
%%	
%%	
%%	
%%	\item \textbf{一级优化:-O1}:\\
%%编译器会尝试减少代码体积和代码运行 时间,但并不执行会花费大量时间的优化操作。 \\
%%对应命令为:
%%\begin{verbatim}
%%	      g++ -O1 -S factorial.i -o factorial_O1.s
%%\end{verbatim}
%%
%%\begin{figure}[H]
%%	\centering
%%	\includegraphics[width=16 CM]{figure/018}
%%	\caption{O1等级优化:factorial\_O1.s部分内容}
%%	\label{O1等级优化:factorial_O1.s部分内容}
%%\end{figure}	
%%相较于-O0,-O1在开头优化了:\\
%%\begin{verbatim}
%%	.local	_ZStL8__ioinit
%%	.comm	_ZStL8__ioinit,1,1
%%\end{verbatim}
%%这两句语句。\\
%%
%%同时-O1在函数栈帧切换时使用了:
%%\begin{verbatim}
%%	subq	$24, %rsp
%%\end{verbatim}
%%而不同于-O0的:
%%\begin{verbatim}
%%	pushq	%rbp
%%	movq	%rsp, %rbp
%%	subq	$32, %rsp
%%\end{verbatim}
%%这种处理减少了入栈出栈过程,同时也节省了函数栈空间的使用。\\
%%此外,-O1模式下,局部变量的值直接使用寄存器进行存储,也能够提升运行效率。\\
%%
%%此外,与没有使用任何优化的-O0选项相比,从-O1等级开始,编译器就会进行\textbf {常量折叠}的有关优化。所谓“常量折叠”,就是在编译器 \textbf {进行语法分析的时候,将常量表达式计算求值,并用求 得的值来替换表达式,放入常量表}。计算时编译器\textbf {直接从表中取值而不用访问内 存,省去了访问内存的时间}。这也是一种编译期优化。 
%%
%%
%%
%%
%%	\item \textbf{二级优化:-O2}:\\
%%-O2 会比-O1 启用多一些标记,是比较推荐的优化等级。设置了-O2 后,编 译器会试图提高代码性能而不会增大体积和大量占用的编译时间。 \\
%%对应命令为:
%%\begin{verbatim}
%%	      g++ -O2 -S factorial.i -o factorial_O2.s
%%\end{verbatim}
%%
%%\begin{figure}[H]
%%	\centering
%%	\includegraphics[width=16 CM]{figure/019}
%%	\caption{O2与O1优化对比:factorial\_O1.s与factorial\_O2.s部分内容}
%%	\label{O2与O1优化对比:factorial_O1.s与factorial_O2.s部分内容}
%%\end{figure}
%%可以发现,在O1级优化的基础上,O2级优化会减少循环的展开次数。\\
%%例如在factorial\_O2.s中就没有出现L4标签,通过调整汇编语句的执行顺序和增删指令,对应高级语言“循环”操作的汇编语言“跳转”次数能够得到减少。
%%
%%	\item \textbf{三级优化:-O3}:\\
%%O3级优化主要针对程序空间大小进行优化。这个选项也会给代码编译时间带来更长的延长。  \\
%%对应命令为:
%%\begin{verbatim}
%%	      g++ -O3 -S factorial.i -o factorial_O3.s
%%\end{verbatim}
%%\begin{figure}[H]
%%	\centering
%%	\includegraphics[width=16 CM]{figure/020}
%%	\caption{O3等级优化:factorial\_O3.s部分内容}
%%\label{O3等级优化:factorial_O3.s部分内容}
%%\end{figure}
%%可以发现,经过O3级优化后,汇编源程序代码数量大幅度提升。
%%
%%	\item \textbf{总结}:\\
%%		
%%\begin{figure}[H]
%%	\centering
%%	\includegraphics[width=16 CM]{figure/021}
%%	\caption{不同优化等级生成的汇编源文件大小比较}
%%	\label{不同优化等级生成的汇编源文件大小比较}
%%\end{figure}	 
%%可以看到,针对O1、O2、O3三种存在优化的优化等级而言,优化级别越高, 虽然最后生成的代码的执行效率越高,但是代码量也会越大,同时编译过程花费的时间也就会越长。 所以我们需要在执行效率和编译时间之间做出一个权衡。
%%
%%\end{itemize}
%	 
%
%
%
%
%
%
