%
%
%\section{汇\quad 编}
%\subsection{汇编过程概述}
%汇编(assembling)是把汇编语言代码翻译成目标机器指令的过程。\\
%
%\subsection{代码段与数据段}
%汇编阶段生成的 文件为二进制文件。目标文件由段组成,通常一个目标文件中至少有两个段:代码段和数据段。\\
%\begin{itemize}
%	\item \textbf{代码段}:代码段中包含的主要是程序的指令,代码段一般是可读可执行的,但一 般不可写。 
%	
%	\item \textbf{数据段}:数据段主要存放程序中要用到的各种全局变量或静态的数据。 数据段一般都是可读,可写,可执行的。
%\end{itemize}
%
%\subsection{例程输出分析}
%
%\subsubsection{汇编器调用命令}
%Linux中,能让GCC编译器使用编译器的输出文件test.s生成目标机器指令文件test.o的命令是:
%
%\begin{verbatim}
%      gcc test.s -c -o test.o
%\end{verbatim}
%
%其中的-c 选项表示只编译而不进行链接。
%
%\subsubsection{例程输出结果}
%
%test.s经过汇编之后得到test.o,执行过程及部分文件内容如下所示:
%
%%\begin{figure}[H]
%%	\centering
%%	\includegraphics[width=16 CM]{figure/022}
%%	\caption{汇编factorial.s生成factorial.o}
%%	\label{fig:汇编factorial.s生成factorial.o}
%%\end{figure}
%
%%\begin{figure}[H]
%%	\centering
%%	\includegraphics[width=16 CM]{figure/0172}
%%	\caption{使用UltraEdit查看test.o文件部分内容}
%%	\label{fig:使用UltraEdit查看test.o文件部分内容}
%%\end{figure}
%%通过 file 命令 可以查看factorial.o的文件类型:
%%\begin{verbatim}
%%      file factorial.o
%%\end{verbatim}
%%\begin{figure}[H]
%%	\centering
%%	\includegraphics[width=16 CM]{figure/024}
%%	\caption{factorial.o文件类型}
%%	\label{fig:factorial.o文件类型}
%%\end{figure}