%\section{链\quad 接}
%\subsection{链接过程概述}
%链接(linking)是处理可重定位文件,把它们的各种符号 引用和符号定义转换为可执行文件中的合适信息的过程。而重定位是将符 号引用与符号定义进行链接的过程。\\
%
%由汇编程序生成的目标文件并不能立即就可以被执行的,其中还有许多没有 解决的问题。例如,某个源文件中的函数可能引用了另一个源文件中定义的某个变量符号,或者可能调用了某个库文件中的函数,等等。链接器的主要工作就是将有关的目标文件彼此连接,将在一个文件中引 用的符号同该符号在另外一个文件中的定义连接起来。 \\
%
%\subsection{静态链接与动态链接}
%链接处理可分为两种:静态链接和动态链接。
%\begin{itemize}
%	\item \textbf{静态链接}:静态链接过程主要是把可重定位文件依 次读入,分析各个文件的文件头,进而依次读 入各个文件的节区,并计算各个节区的虚拟内 存位置。之后,利用计算出的存储位置,对一些需要重定位的符号进行处理, 设定它们的虚拟内存地址等,最终产生一个可执行文件或者是动态链接库。 
%	
%	\item \textbf{动态链接}:动态链接所作的只是在最终的可执行程序中,记录下少量 的信息。在此可执行文件被执行时,动态链接库的全部内容将被映射虚地址空间中。动态链接程序将根据可执行程序中之前记录下来的信息找到相 应的函数指针的地址,进而能够调用这些函数,完成执行过程。 
%\end{itemize}
%
%\subsection{例程输出分析}
%
%\subsubsection{链接器调用命令}
%调用链接器生成test可执行文件的命令是:
%
%\begin{verbatim}
%      gcc test.o -o test
%\end{verbatim}
%
%\subsubsection{例程输出结果}
%
%test.o经过汇编之后得到可执行文件test,链接过程如下所示:
%
%%\begin{figure}[H]
%%	\centering
%%	\includegraphics[width=16 CM]{figure/025}
%%	\caption{链接factorial.o生成factorial}
%%	\label{链接factorial.o生成factorial}
%%\end{figure}
%
%使用如下指令可以运行test可执行文件,检验编译结果:
%\begin{verbatim}
%      ./test
%\end{verbatim}
%
%
%\begin{figure}[H]
%	\centering
%	\includegraphics[width=16 CM]{figure/0252}
%	\caption{运行test可执行文件}
%	\label{fig:运行test可执行文件}
%\end{figure}
