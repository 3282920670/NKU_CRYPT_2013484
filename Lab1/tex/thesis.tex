\documentclass[cs4size,a4paper]{ctexart}   
%==================== 数学符号公式 ============
\usepackage{amsmath}                 % AMS LaTeX宏包
\usepackage[style=1]{mdframed}
\usepackage{amsthm}
\usepackage{amsfonts}
\usepackage{mathrsfs}                % 英文花体字 体
\usepackage{bm}                      % 数学公式中的黑斜体
\usepackage{bbding,manfnt}           % 一些图标,如 \dbend
\usepackage{lettrine}                % 首字下沉,命令\lettrine
\def\attention{\lettrine[lines=2,lraise=0,nindent=0em]{\large\textdbend\hspace{1mm}}{}}
\usepackage{longtable}
\usepackage[toc,page]{appendix}
\usepackage{geometry}                % 页边距调整
\geometry{top=3.0cm,bottom=2.7cm,left=2.5cm,right=2.5cm}
%====================公式按章编号==========================
\numberwithin{equation}{section}
\numberwithin{table}{section}
\numberwithin{figure}{section}
%================= 基本格式预置 ===========================
\usepackage{fancyhdr}
\pagestyle{fancy}
\fancyhf{}  
\fancyhead[C]{\zihao{5}  \kaishu 密码学报告}
\fancyfoot[C]{~\zihao{5} \thepage~}
\renewcommand{\headrulewidth}{0.65pt} 
\CTEXsetup[format={\centering\bfseries\zihao{-2}},name={第, 节}]{section}
\CTEXsetup[nameformat={\bfseries\zihao{3}}]{subsection}
\CTEXsetup[nameformat={\bfseries\zihao{4}}]{subsubsection}
%================== 图形支持宏包 =========================
\usepackage{subfigure}
\usepackage{graphicx}                % 嵌入png图像
\usepackage{color,xcolor}            % 支持彩色文本、底色、文本框等
\usepackage{hyperref}                % 交叉引用
\usepackage{caption}
\captionsetup{figurewithin=section}
%==================== 源码和流程图 =====================
\usepackage{listings}                % 粘贴源代码
\usepackage{xcolor}
\usepackage{color}
\definecolor{dkgreen}{rgb}{0,0.6,0}
\definecolor{gray}{rgb}{0.5,0.5,0.5}
\definecolor{mauve}{rgb}{0.58,0,0.82}
\usepackage{xcolor}
\lstset{
	%行号
	numbers=left,
	%背景框
	framexleftmargin=8mm,
	frame=none,
	%背景色
	%backgroundcolor=\color[rgb]{1,1,0.76},
	backgroundcolor=\color[RGB]{245,245,244},
	%样式
	keywordstyle=\bf\color{blue},
	identifierstyle=\bf,
	numberstyle=\color[RGB]{0,192,192},
	commentstyle=\it\color[RGB]{0,96,96},
	stringstyle=\rmfamily\slshape\color[RGB]{128,0,0},
	%显示空格
	showstringspaces=false
}
%====================自行添加=====================
\definecolor{CPPLight}  {HTML} {686868}
\definecolor{CPPSteel}  {HTML} {888888}
\definecolor{CPPDark}   {HTML} {262626}
\definecolor{CPPBlue}   {HTML} {4172A3}
\definecolor{CPPGreen}  {HTML} {487818}
\definecolor{CPPBrown}  {HTML} {A07040}
\definecolor{CPPRed}    {HTML} {AD4D3A}
\definecolor{CPPViolet} {HTML} {7040A0}
\definecolor{CPPGray}  {HTML} {B8B8B8}


\lstset{
	columns=fixed,       
	numbers=left,                                        % 在左侧显示行号
	frame=none,                                          % 不显示背景边框
	backgroundcolor=\color[RGB]{245,245,244},            % 设定背景颜色
	keywordstyle=\color[RGB]{40,40,255},                 % 设定关键字颜色
	numberstyle=\footnotesize\color{darkgray},           % 设定行号格式
	commentstyle=\it\color[RGB]{0,96,96},                % 设置代码注释的格式
	stringstyle=\rmfamily\slshape\color[RGB]{128,0,0},   % 设置字符串格式
	showstringspaces=false,                              % 不显示字符串中的空格
	language=c++,                                        % 设置语言
}

\lstset{
	columns=fixed,       
	numbers=left,                                        % 在左侧显示行号
	frame=none,                                          % 不显示背景边框
	backgroundcolor=\color[RGB]{245,245,244},            % 设定背景颜色
	keywordstyle=\color[RGB]{40,40,255},                 % 设定关键字颜色
	numberstyle=\footnotesize\color{darkgray},           % 设定行号格式
	commentstyle=\it\color[RGB]{0,96,96},                % 设置代码注释的格式
	stringstyle=\rmfamily\slshape\color[RGB]{128,0,0},   % 设置字符串格式
	showstringspaces=false,                              % 不显示字符串中的空格
	language=c++,                                        % 设置语言
	morekeywords={alignas,continute,friend,register,true,alignof,decltype,goto,
		reinterpret_cast,try,asm,defult,if,return,typedef,auto,delete,inline,short,
		typeid,bool,do,int,signed,typename,break,double,long,sizeof,union,case,
		dynamic_cast,mutable,static,unsigned,catch,else,namespace,static_assert,using,
		char,enum,new,static_cast,virtual,char16_t,char32_t,explict,noexcept,struct,
		void,export,nullptr,switch,volatile,class,extern,operator,template,wchar_t,
		const,false,private,this,while,constexpr,float,protected,thread_local,
		const_cast,for,public,throw,std},
	emph={map,set,multimap,multiset,unordered_map,unordered_set,
		unordered_multiset,unordered_multimap,vector,string,list,deque,
		array,stack,forwared_list,iostream,memory,shared_ptr,unique_ptr,
		random,bitset,ostream,istream,cout,cin,endl,move,default_random_engine,
		uniform_int_distribution,iterator,algorithm,functional,bing,numeric,},
	emphstyle=\color{CPPViolet}, 
}
%====================自行添加=====================



%--------------------
\hypersetup{hidelinks}
\usepackage{booktabs}  
\usepackage{shorttoc}
\usepackage{tabu,tikz}
\usepackage{float}

\usepackage{multirow}



\tabcolsep=1ex
\tabulinesep=\tabcolsep
\newlength\tikzboxwidth
\newlength\tikzboxheight
\newcommand\tikzbox[1]{%
	\settowidth\tikzboxwidth{#1}%
	\settoheight\tikzboxheight{#1}%
	\begin{tikzpicture}
		\path[use as bounding box]
		(-0.5\tikzboxwidth,-0.5\tikzboxheight)rectangle
		(0.5\tikzboxwidth,0.5\tikzboxheight);
		\node[inner sep=\tabcolsep+0.5\arrayrulewidth,line width=0.5mm,draw=black]
		at(0,0){#1};
	\end{tikzpicture}%
}

\makeatletter
\def\hlinew#1{%
	\noalign{\ifnum0=`}\fi\hrule \@height #1 \futurelet
	\reserved@a\@xhline}

\newcommand{\tabincell}[2]{\begin{tabular}{@{}#1@{}}#2\end{tabular}}%

\usepackage{subfigure}

\usepackage{CJK}
\usepackage{ifthen}


\usepackage{graphicx} 
\newcommand{\HRule}{\rule{\linewidth}{0.5mm}}

\newtheorem{Theorem}{定理}
\newtheorem{Lemma}{引理} 
%%使得公式随章节自动编号
\makeatletter
\@addtoreset{equation}{section}
\makeatother
\renewcommand{\theequation}{\arabic{section}.\arabic{equation}}

%-------------------------

\usepackage{pythonhighlight}
\usepackage{tikz}                    
\usepackage{tikz-3dplot}
\usetikzlibrary{shapes,arrows,positioning}
%===================   正文开始    ===================
\begin{document}
	\bibliographystyle{gbt7714-2005}     %论文引用格式
	%===================  定理类环境定义 ===================
	\newtheorem{example}{例}              % 整体编号
	\newtheorem{algorithm}{算法}
	\newtheorem{theorem}{定理}            % 按 section 编号
	\newtheorem{definition}{定义}
	\newtheorem{axiom}{公理}
	\newtheorem{property}{性质}
	\newtheorem{proposition}{命题}
	\newtheorem{lemma}{引理}
	\newtheorem{corollary}{推论}
	\newtheorem{remark}{注解}
	\newtheorem{condition}{条件}
	\newtheorem{conclusion}{结论}
	\newtheorem{assumption}{假设}
	%==================重定义 ===================
	\renewcommand{\contentsname}{目\quad 录}     
	\renewcommand{\abstractname}{摘\quad 要} 
	\renewcommand{\refname}{参考文献}     
	\renewcommand{\indexname}{索引}
	\renewcommand{\figurename}{图}
	\renewcommand{\tablename}{表}
	\renewcommand{\appendixname}{附录}
	\renewcommand{\proofname}{证明}
	\renewcommand{\algorithm}{算法} 
	%============== 封皮和前言 =================
	\input{body/cover}
	\pagestyle{plain}
	\pagenumbering{Roman}
	\section*{\zihao{2} \centering 摘\quad 要}

\vskip0.5cm
%今天,纷繁复杂又无处不在的软件架构维系着全球社会的稳定与发展,而编译技术和高级编程语言正是这些软件的基石。强大而优雅的编译技术,搭起了高级语言与机器语言之间的桥梁,让高级语言能带给程序员更多的便利。同时,了解较为底层的编译器工作原理,也将有助于编程技巧的提升。\\
本文C++语言实现了DES算法,并检测计算了雪崩效应\\


\textbf{关键词:}  C++,DES,雪崩效应
\addcontentsline{toc}{section}{摘要}

\clearpage
%\section*{\zihao{2} \centering \textbf{Abstract} }
   %用了Times New Roman字体来美化观感

%Today, the complex and ubiquitous software architecture maintains the stability and development of the international society. Compiler technology and high-level programming languages are the cornerstones of these software. Powerful and elegant compilation technology bridges the gap between high-level language and machine language, allowing high-level languages to bring more convenience to programmers. At the same time, understanding the working principle of the compiler will also help improve the programming skills. This paper takes the process of compiling C++ language in GCC compiler under Linux environment as an example, combining the literature and experimental results, to analyze the main workflow of the compiler, in order to have a deeper understanding of the working principle of the compiler. \\
%This paper takes the process of GCC compiler compiling C language program under Linux environment as an example, combining literature and experimental results, analyzes the main work flow of the compiler, and analyzes the annotation of LLVM IR programming program.\\
%\textbf{Key Words:} GCC Compiler, Language Processing, Preprocessing, Compiling, Assembling, Linking, LLVM IR programmingh
%\addcontentsline{toc}{section}{Abstract}





	\pagestyle{empty}
	\tableofcontents %这句话就能自动把目录都加进来
	\thispagestyle{empty}
	%============== 论文正文   =================
	\pagestyle{fancy}
	
\section{移位密码的加解密}

\subsection{移位密码的加解密流程}
\begin{figure}[thbp!]
	\centering
	\includegraphics[width=16cm]{figure/figure1.png}
	\caption{移位密码的加解密流程}
	\label{fig:移位密码的加解密流程}
\end{figure}

\subsection{移位密码程序代码}
\begin{lstlisting}[language=c++]
#include <iostream>
using namespace std;

class shift_crypt
{
	private:
	int offset;//移位
	char* ciphertext;//密文
	public:
	shift_crypt()
	{
		offset = 0;
	}
	shift_crypt(int offset)
	{
		this->offset = offset;
	}
	char* get_ciphertext()
	{
		return this->ciphertext;
	}
	void shift_encrypt(char* plaintext) //加密
	{
		offset = offset % 26;
		//cout << "offset: " << offset << endl;
		int len = strlen(plaintext);
		//cout << "len: " << len << endl;
		this->ciphertext = new char[len];
		for (int i = 0; i < len; i++)
		{
			//cout << "plaintext[i]: " << plaintext[i] << endl;
			if (plaintext[i] >= 'a' && plaintext[i] <= 'z' || plaintext[i] >= 'A' && plaintext[i] <= 'Z')
			{
				char temp = plaintext[i] + offset;
				//cout << "temp: " << temp << endl;
				if (temp > 'Z' && plaintext[i]<='Z' || temp > 'z')
				this->ciphertext[i] = temp - 26;
				else
				this->ciphertext[i] = temp;
			}
			else
			{
				this->ciphertext[i] = plaintext[i];
			}
			//cout << "ciphertext[i]: " << ciphertext[i] << endl;
		}
		this->ciphertext[len] = '\0';
	}
	char* shift_decrypt(char* ciphertext, int offset)//解密
	{
		int len = strlen(ciphertext);
		char* plaintext = new char[len];
		for (int i = 0; i < len; i++)
		{
			if (ciphertext[i] >= 'a' && ciphertext[i] <= 'z' || ciphertext[i] >= 'A' && ciphertext[i] <= 'Z')
			{
				char temp = ciphertext[i] - offset;
				if (temp < 'a' && ciphertext[i] >= 'a' || temp < 'A')
				plaintext[i] = temp + 26;
				else
				plaintext[i] = temp;
			}
			else
			plaintext[i] = ciphertext[i];
		}
		plaintext[len] = '\0';
		return plaintext;
	}
	void exhaust_decrypt(char* ciphertext)//穷举
	{
		int offset;
		char* plaintext;
		for (offset = 0; offset <= 25; offset++)
		{
			plaintext = shift_decrypt(ciphertext, offset);
			cout << "移位为:" << offset << " 时明文为:" << plaintext << endl;
		}
	}
};
int main()
{
	char* plaintext = new char[1024];
	char* ciphertext = new char[1024];
	int offset;
	cout << "请输入移位:";
	cin >> offset;
	cout << "请输入要加密的明文:";
	cin >> plaintext;
	
	shift_crypt pro = shift_crypt(offset);
	pro.shift_encrypt(plaintext);
	char* temp1 = pro.get_ciphertext();
	cout << "移位加密后的密文为:" << temp1 << endl << endl;
	cout << "请输入要解密的密文:";
	cin >> ciphertext;
	char* temp2 = pro.shift_decrypt(ciphertext, offset);
	cout << ciphertext << "对应的明文为:" << temp2 << endl << endl;
	
	cout << "穷举攻击的密文为:"<<temp1 << endl;
	pro.exhaust_decrypt(temp1);
	return 0;
}
\end{lstlisting}

\subsection{程序运行结果}
\begin{figure}[thbp!]
	\centering
	\includegraphics[width=16cm]{figure/002.png}
	\caption{移位密码加解密程序运行结果}
	\label{fig:移位密码加解密程序运行结果}
\end{figure}

%\begin{enumerate}
%	\item \textbf {预处理器:}处理源代码中以\#开始的预编译指令,例如展开所有宏定义、插入\#include指向的文件等,以获得经过预处理的源程序。
%	
%	\item \textbf {编译器:}将预处理器处理过的源程序文件翻译成为标准的汇编语言以供计算机阅读。
%	
%	\item \textbf {汇编器:}将汇编语言指令翻译成机器语言指令,并将汇编语言程序打包成可重定位目标程序。
%	
%	\item \textbf {链接器:}将可重定位的机器代码和相应的一些目标文件以及库文件连接在一起,形成真正能在机器上运行的目标机器代码。
%\end{enumerate}

%\begin{figure}[thbp!]
%	\centering
%	\includegraphics[height=6.8 CM]{figure/001}
%	\caption{语言处理过程图示}
%	\label{fig:语言处理过程图示}
%\end{figure}

























      %
	%\section{编译环境简介}
%在开发应用程序时,高级语言是必不可少的工具。利用高级语言开发自然离不开高级语言编译器,而GCC(GNU Compiler Collection)就是目前Linux下最常用的高级语言编译器。GCC是GNU推出的功能强大的编译器套件,是GNU项目中符合ANSI C标准的编译系统,能够编译使用C、C++、Objective-C等语言编写的程序;同时,GCC也可以在多种硬件平台上编译出可执行程序,且具有较高的执行效率。\\
%“工欲善其事,必先利其器”。本部分对GCC编译系统的相关内容予以介绍。
%
%\subsection{GCC编译器简介}
%GCC是以GPL许可证所发行的自由软件,也是GNU计划的关键部分。GCC的初衷是为GNU操作系统专门编写一款编译器,现已被大多数类Unix操作系统(如Linux、BSD、MacOS X等)采纳为标准的编译器,甚至在微软的Windows上也可以使用GCC。GCC支持多种计算机体系结构芯片,如x86、ARM、MIPS等,并已被移植到其他多种硬件平台。
%GCC原名为GNU C语言编译器(GNU C Compiler),只能处理C语言。但其很快扩展,变得可处理C++,后来又扩展为能够支持更多编程语言,如Fortran、Pascal、Objective -C、Java、Ada、Go以及各类处理器架构上的汇编语言等,所以改名GNU编译器套件(GNU Compiler Collection)。\\
%
%GCC允许程序员将编译过程中得到的语法中间表示导出为数据文件,程序员可以通过该中间文件直接获取 GCC 编译代码过程中的内部信息。本报告所要研究的是编译器在编译程序过程中的工作流程,因此通过制定编译选项来查看产生的中间文件,可以大大降低我们的分析难度。因此,在本次报告中,我们以Linux环境下GCC编译器编译factorial.cpp的过程作为分析样例,来探究\textbf {预处理器}、\textbf {编译器}、\textbf {汇编器}、\textbf {链接器}四个部分在程序编译过程中发挥的功能。
%
%
%%\subsection{gcc命令与g++命令的介绍与比较}
%
%%\subsubsection{gcc与g++简介}
%%前文已经介绍过,GCC是指GUN 编译器套件(GNU Compiler Collection),它可以编译C、C++、JAVA、Fortran、Pascal、Object-C、Ada等多种语言。\\
%%
%%而gcc是GCC套件中的C编译器,即GUN C Compiler;g++则是GCC套件中的C++编译器,即GUN C++ Compiler。\\
%%
%%更深一步来讲,gcc和g++本质上并不是编译器,更不是编译器集合,而只是一种驱动器。根据参数中要编译的文件的类型,gcc和g++会调用对应的GUN编译器进行编译。所以,更准确的说法是:gcc调用了C Compiler,而g++调用了C++ Compiler\\
%%
%%既然gcc和g++只负责调用编译器的驱动器,而不是直接编译C和C++程序的编译器,那么也就不难理解这样一个事实:gcc和g++都是可以处理C语言源程序和C++语言源程序的。
%%
%%
%%其实,g++处理程序时,在编译阶段,g++驱动器会调用gcc驱动器,由gcc按照处理C++程序的方式进行处理,即编译工作最终都是由gcc来完成的;而链接阶段的工作则由g++自己来完成,因为gcc命令不能自动和c++程序使用的库连接,而g++才能自动调用链接的c++库。这一特性也导致两者在处理源程序时是有所区别的。下文对两者的主要区别进行阐释。
%%
%%\subsubsection{gcc命令与g++命令比较}
%%gcc和g++的主要区别有以下几点:
%%
%%\begin{itemize}
%%	\item gcc把.c后缀的文件\textbf{当做是C程序},把.cpp后缀的文件\textbf{当做是C++程序};g++把.c后缀和.cpp后缀的文件\textbf{都统一当做是C++程序}。
%%	
%%	\item 使用g++处理文件时,g++\textbf{会自动链接标准库STL};使用gcc处理文件时,gcc则\textbf{不会自动链接标准库STL}。因此,使用gcc处理c++文件时,为了能够使用STL,需要加参数 "–lstdc++"来手动链接C++库,指令形如"gcc factorial.cpp -lstdc++"。
%%	
%%	\item gcc在处理.c后缀文件时,可使用的预定义宏是比较少的;gcc在处理.cpp后缀文件以及g++在处理.c后缀文件和.cpp后缀文件时,会加入一些额外的宏。涉及gcc与g++差异的有关宏的定义参见附录1。	
%%\end{itemize}
%
%
%\subsection{GCC编译器基本用法}
%
%使用gcc指令编译程序的基本指令格式是:
%\begin{verbatim}
%      gcc  [options] filenames
%\end{verbatim}
%
%其中的参数含义如下:
%\begin{verbatim}
%      options :编译器所需要的编译选项,为可选参数,可以没有
%      filenames :要编译的文件名
%\end{verbatim}
%
%同时,通过设置编译选项,我们可以获得GCC编译器在编译过程中生成的中间文件。下面以编译factorial.cpp源程序为例,展示相应使用到的基本gcc指令。
%
%%\begin{figure}[thbp!]
%%	\centering
%%	\includegraphics[height=6.8 CM]{figure/002}
%%	\caption{gcc基本指令示意}
%%	\label{fig:gcc基本指令示意}
%%\end{figure}
%
%%\begin{figure}[thbp!]
%%	\centering
%%	\includegraphics[height=10 CM]{figure/003}
%%	\caption{gcc中间结果文件关系示意}
%%	\label{fig:gcc中间结果文件关系示意}
%%\end{figure}
%
%%\subsection{实验环境信息}
%%本报告中,我们以用C++语言编写的阶乘源程序factorial.cpp为例,分析其经过预处理得到的中间文件factorial.i。其中,所使用的Linux发行版操作系统版本为Ubuntu 16.04.5,所使用的GCC编译器版本为GCC 5.4.0 20160609,如下图所示:\\
%%
%%\begin{figure}[thbp!]
%%	\centering
%%	\includegraphics[height=7 CM,width=16 CM]{figure/004}
%%	\caption{操作系统及编译器版本}
%%	\label{fig:操作系统及编译器版本}
%%\end{figure}
%%
%%需要说明的是,本实验中使用的例程为C++语言编写的程序。为了能得到完整的包括C++库链接在内的编译过程,本次实验使用了g++指令。如前文所述,g++与gcc均属于GCC的驱动器,所以两者对应的编译器版本是完全相同的,均为GCC 5.4.0 20160609,如下图所示:
%%
%%\begin{figure}[htbp!]
%%	\includegraphics[height=7 CM,width=16 CM]{figure/005}
%%	\caption{gcc与g++指令对应编译器版本信息比较}
%%	\label{fig:gcc与g++指令对应编译器版本信息比较}
%%\end{figure}
%
%
%
%
%
%
%
%
%
%
%
%
%
%
%
%
%
%
%
%
%
%
%

	%
%
%\section{预处理}
%\subsection{预处理过程概述}
%预处理阶段会处理预编译指令,包括绝大多数的 \# 开头的指令,如 include define if 等等,对
%include 指令会替换对应的头文件,对 define 的宏命令会直接替换相应内容,同时会删除注释,添
%加行号和文件名标识。\\
%对于 gcc,通过添加参数-E 令 gcc 只进行预处理过程,参数-o 改变 gcc 输出文件名,因此通过命
%令 gcc main.c -E -o main.i,即可得到预处理后文件。\\
%观察预处理文件,可以发现文件长度远大于源文件,这就是将代码中的头文件进行了替代导致的
%结果。另外,实际上预处理过程是 gcc 调用了另一个程序(C Pre-Processor 调用时简写作 cpp)完成
%的过程,有兴趣的同学可以自行尝试。\\
%
%%\subsection{预处理的目的}
%%预处理的功能主要是对源程序进行一些文本层面的处理。之所以要设置预处理环节,是因为预处理能够方便代码的编写,也能提高程序的运行效率。\\
%%
%%以程序中的宏定义为例。一方面,通过在程序中宏定义需要的常量,我们在编程时对于程序的修改将方便很多。另一方面,在函数调用时,使用带参数的宏定义完成参数传递,可以减少系统开销,提高运行效率。这是因为如果在预处理阶段即进行了宏展开,那么程序在执行时就不需要再去转换,在当地执行即可取出参数的值。\\
%
%\subsection{预处理语句}
%\subsubsection{预处理语句的功能}
%除了预先定义的宏之外,预处理器所有的功能均由预处理语句触发。下面依照GCC官方文档,对源程序中预处理语句的功能予以罗列:
%
%\begin{itemize}
%	\item \textbf{包含头文件}(Inclusion of header files):这些是可以替换到您的程序中的声明文件。
%	
%	\item \textbf{宏展开}(Macro expansion):您可以定义宏,它们是C代码
%	的任意片段的缩写。预处理器将在整个程
%	序中用它们的定义代替宏。某些宏会自动
%	为您定义。
%	
%	\item \textbf{条件编译}(Conditional compilation):您可以根据各种条件引入或去除部分程序。
%	
%	\item \textbf{行控制}(Line control):如果您想使用您的程序把一些源文件组合起来或者重新编排,将其生成一个中间文件,之后再去编译这个中间文件的话,您可以使用行控制来告诉编译器该中间文件的每一行来自哪里。	
%	
%	\item \textbf{诊断}(Diagnostics):您可以在编译时检测问题,并发出错误或警告。		
%	
%\end{itemize}
%
%\subsubsection{常用预处理命令}
%预处理命令由\#开头,它独占一行,\#之前只能是空白符。在C/C++语言源文件中,以\#开头的语句就是预处理语句,不以\#开头的语句为C/C++中的代码行。常用的预处理命令列举如下:
%\begin{table}[H]
%	\caption{常用预处理命令简介}
%	\centering
%	\begin{tabular}{l|l}
%		\toprule
%		预处理命令 &  \quad \quad \quad \quad \quad \quad \quad \quad \quad \quad 对应功能 \\
%		\midrule[1.5pt]
%		\#define & 定义一个预处理宏   \\
%		\#undef & 取消宏的定义  \\
%		\#include & 包含一个文件   \\
%		\#if & 预处理语法中的条件命令,
%		相当于C语法中的 if 语句   \\
%		
%		\#ifdef & 判断某个宏是否被定义,
%		若已定义,执行随后的语句   \\
%		
%		\#ifndef & 判断某个宏是否未被定义,
%		若未定义,执行随后的语句   \\
%		
%		\# else & 与\#if,\#ifdef,\#ifndef 对应,若这些条件不满足,\\
%		
%		 & 则执行\#else 之后的语句,相当于C语法中的else\\
%
%		\# elif & 若\#if,\#ifdef,\#ifndef 或之前的\#elif条件不满足,\\
%		
%		 & 则执行\#elif 之后的语句,相当于C语法中的 else-if\\		 
%		 
%		\#endif & \#if,\#ifdef,\#ifndef 这些条件命令的结束标志   \\
%		
%		
%		\#pragma & 说明编译器信息   \\
%		\#warning & 显示编译警告信息   \\
%		\#error & 显示编译错误信息   \\
%		
%		\bottomrule
%	\end{tabular}
%\end{table}
%
%%\subsubsection{预处理语句的文法}
%%预处理并不分析整个源代码文件,它只是将源代码分割成一些标记(token),识别语句中哪些是 C语句,哪些是预处理语句。\\
%%
%%预处理语句的一般格式如下:
%%
%%\begin{verbatim}
%%      #command name(...) token(s) 
%%\end{verbatim}
%%
%%其中的参数含义如下:
%%\begin{verbatim}
%%      command:预处理命令的名称,它之前以#开头,#之后紧随预处理命令,通常不能有空格。 若某行中只包含#以及空白符,那么在C语言中该行被理解为空白。
%%      name:代表宏名称,它可带参数,参数可以是可变参数列表。
%%      \:语句中可以利用"\"来换行。
%%\end{verbatim}
%
%\subsection{例程输出分析}
%
%\subsubsection{预处理器调用命令}
%Linux中使用gcc驱动器调用预处理器,处理源文件test.c得到预处理输出文件test.i的命令是:
%
%\begin{verbatim}
%      gcc test.c -E -o test.i
%\end{verbatim}
%
%其中的-E参数可以让GCC只对源代码进行预处理而不进行后续编译操作,并将与处理之后的代码输出到命令中指定的输出文件中。
%
%\subsubsection{例程输出结果}
%
%test.c经过预处理之后得到test.i,执行过程及部分文件内容如下所示:
%
%%\begin{figure}[H]
%%	\centering
%%	\includegraphics[width=16 CM]{figure/0062}
%%	\caption{预处理源程序test.c}
%%	\label{fig:预处理源程序test.c}
%%\end{figure}
%
%%\begin{figure}[H]
%%	\centering
%%	\includegraphics[width=16 CM]{figure/0062}
%%	\caption{test.c文件内容}
%%	\label{fig:test.c文件内容}
%%\end{figure}
%
%%\begin{figure}[H]
%%	\centering
%%	\includegraphics[width=16 CM]{figure/0082}
%%	\caption{test.i部分文件内容-1}
%%	\label{fig:test.i部分文件内容-1}
%%\end{figure}
%
%%\begin{figure}[H]
%%	\centering
%%	\includegraphics[width=16 CM]{figure/0092}
%%	\caption{test.i部分文件内容-2}
%%	\label{fig:test.i部分文件内容-2}
%%\end{figure}
%
%
%%此外,还对源程序进行局部修改之后查看了输出结果,相关操作将在下一部分【 例程结果分析】中详细说明。
%
%%\subsubsection{例程输出结果分析}
%%通过上述实验及系列其他实验( 下文会详细说明),我们可以发现,预处理器对于源文件进行了一些处理,替换了一些内容,丢弃了一些内容,保留了一些内容,增加了一些内容:
%%
%%\begin{enumerate}
%%	\item \textbf{替换部分内容}:我们可以发现生成的factorial.i的文件长度远长于文件factorial.cpp的文件长度。这是因为预处理器会处理以\#开头的命令,根据命令的内容进行宏替换、处理条件预处理指令、进行文件包含替换等。例如本实验中源程序的"\#include<iostream>"语句就没有出现在输出结果中,相应的C++头文件被替换进了源文件中。
%%	
%%	\begin{figure}[H]
%%		\centering
%%		\includegraphics[height=6 CM, width=16 CM]{figure/010}
%%		\caption{factorial.i部分文件内容-3}
%%		\label{fig:factorial.i部分文件内容-3}
%%	\end{figure}
%%
%%    除了factorial.cpp涉及的文件包含预处理指令的替换,预处理器还会进行如下替换相关操作:
%%
%%\begin{itemize}
%%	\item \textbf{宏定义替换}:所有宏定义行会被空白行替代,所有使用宏定义的位置会被实际内容替换。\\
%%	
%%	例如在如下实验中,test1-1.cpp中pi和abc()两处宏定义就被预处理器识别并展开,而原始的两处宏定义行被替换成了空白行:
%%	
%%	\begin{figure}[H]
%%	\centering
%%	\includegraphics[height=6 CM]{figure/011}
%%	\caption{test1-1.cpp预处理过程中的宏替换}
%%	\label{test1-1.cpp预处理过程中的宏替换}
%%    \end{figure}	
%%	
%%	\item \textbf{注释替换}:所有注释都将被空格或者空行替代。\\
%%	
%%	例如在如下实验中,test1-2.cpp中两处注释就被替换成了空格或者空行:	
%%	\begin{figure}[H]
%%	\centering
%%	\includegraphics[height=7 CM]{figure/012}
%%	\caption{test1-2.cpp预处理过程中的注释替换}
%%	\label{test1-2.cpp预处理过程中的注释替换}
%%    \end{figure}
%%
%%\end{itemize}
%%
%%	\item \textbf{丢弃空行空格}:空白的长行将被丢弃。根据ISO标准规定,在GNU CPP中,空白的行会被丢弃,参数之间的空格被折叠成为单个空格。
%%
%%	\item \textbf{保留布局控制指令}:所有\#pragma布局控制预处理语句会被保留,因为编译器会使用到它们。
%%	
%%	\item \textbf{增加行标}: 
%%	
%%	观察factorial.i可以发现其中多出了许多形如以下的代码:
%%	
%%	\begin{figure}[H]
%%	\centering
%%	\includegraphics[height=7 CM]{figure/013}
%%	\caption{factorial.i中的行标}
%%	\label{factorial.i中的行标}
%%    \end{figure}	
%%	
%%	这些被称为linemarkers(行标),其格式遵循:
%%	
%%	\begin{verbatim}
%%	      # linenum filename flags 
%%	\end{verbatim}
%%	
%%	其中:
%%	\begin{verbatim}
%%	语句中的linenum是指filename中相应文件的对应行。 若某行中只包含#以及空
%%	白符,那么在C语言中该行被理解为空白。
%%	
%%	flags含义如下:
%%	     1:表示引入了一个新文件
%%	     2:表示返回了一个文件(该文件是包含了其他文件之后的文件)
%%	     3:表示以下文本来自系统头文件,因此应该抑制某些警告
%%	     4:表示以下文本应被视为包裹在隐式extern "C"块中    
%%	
%%	\end{verbatim}
%%	
%%	\item \textbf{处理条件编译}:预处理器将根据\#if,\#ifdef,\#ifndef,\#endif 来确定是否需要对各部分进行相应处理。使用条件编译可以使目标程序变小,运行时间变短。\\
%%	
%%	程序test1-3.cpp预处理结果如下:
%%	
%%	\begin{figure}[H]
%%	\centering
%%	\includegraphics[height=7 CM]{figure/014}
%%	\caption{test1-3.cpp预处理过程中的条件编译处理}
%%	\label{test1-3.cpp预处理过程中的条件编译处理}
%%    \end{figure}	
%%
%%
%%
%%	
%%\end{enumerate}
%%
%
%
%

	%
%
%\section{编\quad 译}
%\subsection{编译过程概述}
%编译(compiling)就是通过词法分析和语法分析,在确认所有指令都符合语法规则之后,将其翻译成等价的中间代码或者是汇编代码的过程。\\
%
%编译器是GCC的核心部件,其功能的实现过程也相对复杂。大体来说,GCC编译器首先检查代码是否有语法错误,确认代码无误后,GCC才会继续进行使用预处理器的输出文件生成汇编源文件。\\
%
%
%
%%\subsection{AT\&T格式汇编语言}
%%GCC编译器的输出结果是汇编代码。既然想要对编译过程进行分析,就要了解GCC编译器输出的汇编语言语法。\\
%%
%%Linux内核代码大量使用内嵌汇编,以进行某些特定功能的实现,或对某功能进行快速实现。Linux 系统中使用的汇编语言格式为AT\&T;而我们在以前只接触过Intel格式汇编语言和MIPS格式汇编语言。尽管宏观理念是大体类似的,但是细节处的语法差异对于我们理解程序也有着较大的困阻。\\
%%
%%因此,本部分对AT\&T风格汇编语言和Intel风格汇编语言的主要语法差异进行归纳总结,以期能通过比照学习,尽快熟悉AT\&T汇编语言。
%%
%%\begin{enumerate}
%%	\item \textbf {字母大小写}:\\
%%	Intel格式的指令使用大写字母,而AT\&T格式的使用小写字母。\\
%%	
%%	\item \textbf {操作数赋值方向}:\\
%%	在Intel语法中,第一个表示目的操作数,第二个表示源操作数,赋值方向从右向左;
%%	而在AT\&T语法中,第一个为源操作数,第二个为目的操作数,方向从左到右,合乎自然。\\
%%	\textbf {示例:将ebx的值赋给eax}:\\
%%	Intel:MOV EAX,EBX \\
%%	AT\&T:movl \%ebx,\%eax\\
%%	
%%	\item \textbf {前缀}:\\
%%	在Intel语法中寄存器和立即数不需要前缀;\\在AT\&T 中寄存器需要加前缀“\%” ,而立即数需要加前缀“\$” 。\\
%%	\textbf {示例:将1赋值给eax}:\\
%%	Intel:MOV EAX,1 \\
%%	AT\&T:movl \$1,\%eax \\
%%	
%%	
%%	\textbf {示例:子过程调用}:\\
%%	Intel:CALL FAR SECTION:OFFSET \\
%%	AT\&T:lcall \$secion:\$offset  \\	
%%	
%%	\textbf {示例:远程跳转}:\\
%%	Intel:JMP FAR SECTION:OFFSET \\
%%	AT\&T:ljmp \$secion:\$offset  \\	
%%	
%%	\textbf {示例:调用/跳转返回}:\\
%%	Intel:RET FAR SATCK\_ADJUST \\
%%	AT\&T:lret \$stack\_adjust   \\	
%%	
%%	\item \textbf {间接寻址}:\\
%%	Intel中基地址使用“[” 、“]”;而在 AT\&T 中使用“(”、“)” 。\\
%%	另外两者处理复杂操作数的语法也不同,Intel为“Segreg:[base+index*scale+disp]”;而在AT\&T中为“\%segreg:disp(base,index,sale)”,其中segreg,index,scale,disp都是可选的,在指定index而没有显式指定Scale的情况下使用默认值。\\
%%	\textbf {示例:间接寻址}:\\
%%	Intel:INSTR FOO,SEGREG:[BASE+INDEX*SCALE+DISP] \\
%%	AT\&T:instr  \%segreg:disp(base,index,scale),foo 
%%	\\
%%	
%%	\item \textbf {后缀}:\\
%%	AT\&T语法中大部分指令操作码的最后一个字母表示操作数大小, “b”表示 byte(一个字节),“w”表示 word(2 个字节),“l”表示 long(4 个字节);Intel 中处理内存操作数时也有类似的语法,如:BYTE PTR、WORD PTR、DWORD PTR。\\
%%	\textbf {示例:赋值}:\\
%%	Intel:MOV AL, BL \\
%%	AT\&T:movb \%bl,\%al \\
%%
%%	Intel:MOV AX,BX \\
%%	AT\&T:movw \%bx,\%ax \\
%%	
%%	Intel:EAX, DWORD PTR[EBX] \\
%%	AT\&T:movl (\%ebx), \%eax \\
%%	
%%	此外,AT\&T 汇编指令中跳转指令标号后的后缀表示跳转方向, “f” 表示向前 (forward) , “b” 表示向后 (back) 。
%%	\textbf {示例:跳转}:\\
%%	 \quad AT\&T:\\
%%	 \quad XXXXXXX \\
%%	 \quad jmp 1f \\
%%     1: mov \$0x8000C580, \%eax \\
%%
%%\end{enumerate}
%
%\subsection{例程输出分析}
%
%\subsubsection{编译器调用命令}
%Linux中,能让GCC编译器使用预处理器的输出文件test.i生成汇编源文件test.s的命令是:
%
%\begin{verbatim}
%      gcc test.i -S -o test.s
%\end{verbatim}
%
%其中的-S 参数表示 GCC 只生成汇编源代码而不
%进行汇编。\\
%输出的.s 文件——汇编语言源程序,在后
%期阶段不再进行预处理操作,而直接进行汇编操作。\\
%输出的.S 文件——汇编语言源程序,在后
%期阶段还会进行预处理、汇编等操作。
%
%\subsubsection{例程输出结果}
%
%test.i经过编译之后得到test.s,执行过程及部分文件内容如下所示:
%
%%\begin{figure}[H]
%%	\centering
%%	\includegraphics[width=16 CM]{figure/015}
%%	\caption{编译test.i生成test.s}
%%	\label{fig:编译test.i生成test.s}
%%\end{figure}
%
%%\begin{figure}[H]
%%	\centering
%%	\includegraphics[width=16 CM]{figure/0152}
%%	\caption{test.s文件部分内容}
%%	\label{fig:test.s文件部分内容}
%%\end{figure}
%
%%\subsection{编译优化处理}
%%
%%\subsubsection{编译优化概述}
%%编译阶段会对代码进行优化处理,优化处理是编译系统中一项比较艰深的技术。它涉及到的问题不仅同编译技术本身有关,而且同机器的硬件环境也有很大的关系。优化分为两部分:优化一部分是对中间代码的优化,这种优化不依赖于具体的计算机。另一种优化则主要针对目标代码的生成而进行的。
%%
%%\begin{itemize}
%%	\item \textbf{不依赖于计算机硬件结构的优化}:主要是删除 公共表达式、循环优化(  代码外提、强度 削弱、变换循环控制、已知量的合并等)、 无用赋值的删除等。 
%%	
%%	\item \textbf{同计算机硬件结构相关的优化}:主要考虑如 何充分利用机器的硬件寄存器存放的有 关变量的值以减少内存的访问次数;根据 机器硬件执行指令的特点对指令进行调 整使目标代码比较短,执行效率更高等。	
%%\end{itemize}
%%
%%\subsubsection{-O优化选项分析}
%%
%%	 -O 选项可以使编译器对代码进行自动优化编译,输出效率更高的可执行文 件。-O 后面还可以跟上数字指定优化级别,如:-O0、-O1、-O2、-O3 等,其中-O0 这个等级关闭所有的优化选项。没有数字默认是 1,最大可以加到 3,优化级别越高,产生的代码的执行效率就越高,但是编译的过程花费的时间就会越长。
%%	 
%%\begin{itemize}
%%	\item \textbf{关闭优化:-O0}:\\
%%	设置为这个优化等级将关闭所有的优化选项。\\
%%	对应命令为:
%%	\begin{verbatim}
%%	      g++ -O0 -S factorial.i -o factorial_O0.s
%%	\end{verbatim}
%%	
%%\begin{figure}[H]
%%	\centering
%%	\includegraphics[width=16 CM]{figure/017}
%%	\caption{O0等级优化:factorial\_O0.s部分内容}
%%	\label{fig:O0等级优化:factorial_O0.s部分内容}
%%\end{figure}
%%	
%%	
%%	
%%	\item \textbf{一级优化:-O1}:\\
%%编译器会尝试减少代码体积和代码运行 时间,但并不执行会花费大量时间的优化操作。 \\
%%对应命令为:
%%\begin{verbatim}
%%	      g++ -O1 -S factorial.i -o factorial_O1.s
%%\end{verbatim}
%%
%%\begin{figure}[H]
%%	\centering
%%	\includegraphics[width=16 CM]{figure/018}
%%	\caption{O1等级优化:factorial\_O1.s部分内容}
%%	\label{O1等级优化:factorial_O1.s部分内容}
%%\end{figure}	
%%相较于-O0,-O1在开头优化了:\\
%%\begin{verbatim}
%%	.local	_ZStL8__ioinit
%%	.comm	_ZStL8__ioinit,1,1
%%\end{verbatim}
%%这两句语句。\\
%%
%%同时-O1在函数栈帧切换时使用了:
%%\begin{verbatim}
%%	subq	$24, %rsp
%%\end{verbatim}
%%而不同于-O0的:
%%\begin{verbatim}
%%	pushq	%rbp
%%	movq	%rsp, %rbp
%%	subq	$32, %rsp
%%\end{verbatim}
%%这种处理减少了入栈出栈过程,同时也节省了函数栈空间的使用。\\
%%此外,-O1模式下,局部变量的值直接使用寄存器进行存储,也能够提升运行效率。\\
%%
%%此外,与没有使用任何优化的-O0选项相比,从-O1等级开始,编译器就会进行\textbf {常量折叠}的有关优化。所谓“常量折叠”,就是在编译器 \textbf {进行语法分析的时候,将常量表达式计算求值,并用求 得的值来替换表达式,放入常量表}。计算时编译器\textbf {直接从表中取值而不用访问内 存,省去了访问内存的时间}。这也是一种编译期优化。 
%%
%%
%%
%%
%%	\item \textbf{二级优化:-O2}:\\
%%-O2 会比-O1 启用多一些标记,是比较推荐的优化等级。设置了-O2 后,编 译器会试图提高代码性能而不会增大体积和大量占用的编译时间。 \\
%%对应命令为:
%%\begin{verbatim}
%%	      g++ -O2 -S factorial.i -o factorial_O2.s
%%\end{verbatim}
%%
%%\begin{figure}[H]
%%	\centering
%%	\includegraphics[width=16 CM]{figure/019}
%%	\caption{O2与O1优化对比:factorial\_O1.s与factorial\_O2.s部分内容}
%%	\label{O2与O1优化对比:factorial_O1.s与factorial_O2.s部分内容}
%%\end{figure}
%%可以发现,在O1级优化的基础上,O2级优化会减少循环的展开次数。\\
%%例如在factorial\_O2.s中就没有出现L4标签,通过调整汇编语句的执行顺序和增删指令,对应高级语言“循环”操作的汇编语言“跳转”次数能够得到减少。
%%
%%	\item \textbf{三级优化:-O3}:\\
%%O3级优化主要针对程序空间大小进行优化。这个选项也会给代码编译时间带来更长的延长。  \\
%%对应命令为:
%%\begin{verbatim}
%%	      g++ -O3 -S factorial.i -o factorial_O3.s
%%\end{verbatim}
%%\begin{figure}[H]
%%	\centering
%%	\includegraphics[width=16 CM]{figure/020}
%%	\caption{O3等级优化:factorial\_O3.s部分内容}
%%\label{O3等级优化:factorial_O3.s部分内容}
%%\end{figure}
%%可以发现,经过O3级优化后,汇编源程序代码数量大幅度提升。
%%
%%	\item \textbf{总结}:\\
%%		
%%\begin{figure}[H]
%%	\centering
%%	\includegraphics[width=16 CM]{figure/021}
%%	\caption{不同优化等级生成的汇编源文件大小比较}
%%	\label{不同优化等级生成的汇编源文件大小比较}
%%\end{figure}	 
%%可以看到,针对O1、O2、O3三种存在优化的优化等级而言,优化级别越高, 虽然最后生成的代码的执行效率越高,但是代码量也会越大,同时编译过程花费的时间也就会越长。 所以我们需要在执行效率和编译时间之间做出一个权衡。
%%
%%\end{itemize}
%	 
%
%
%
%
%
%

	%
%
%\section{汇\quad 编}
%\subsection{汇编过程概述}
%汇编(assembling)是把汇编语言代码翻译成目标机器指令的过程。\\
%
%\subsection{代码段与数据段}
%汇编阶段生成的 文件为二进制文件。目标文件由段组成,通常一个目标文件中至少有两个段:代码段和数据段。\\
%\begin{itemize}
%	\item \textbf{代码段}:代码段中包含的主要是程序的指令,代码段一般是可读可执行的,但一 般不可写。 
%	
%	\item \textbf{数据段}:数据段主要存放程序中要用到的各种全局变量或静态的数据。 数据段一般都是可读,可写,可执行的。
%\end{itemize}
%
%\subsection{例程输出分析}
%
%\subsubsection{汇编器调用命令}
%Linux中,能让GCC编译器使用编译器的输出文件test.s生成目标机器指令文件test.o的命令是:
%
%\begin{verbatim}
%      gcc test.s -c -o test.o
%\end{verbatim}
%
%其中的-c 选项表示只编译而不进行链接。
%
%\subsubsection{例程输出结果}
%
%test.s经过汇编之后得到test.o,执行过程及部分文件内容如下所示:
%
%%\begin{figure}[H]
%%	\centering
%%	\includegraphics[width=16 CM]{figure/022}
%%	\caption{汇编factorial.s生成factorial.o}
%%	\label{fig:汇编factorial.s生成factorial.o}
%%\end{figure}
%
%\begin{figure}[H]
%	\centering
%	\includegraphics[width=16 CM]{figure/0172}
%	\caption{使用UltraEdit查看test.o文件部分内容}
%	\label{fig:使用UltraEdit查看test.o文件部分内容}
%\end{figure}
%%通过 file 命令 可以查看factorial.o的文件类型:
%%\begin{verbatim}
%%      file factorial.o
%%\end{verbatim}
%%\begin{figure}[H]
%%	\centering
%%	\includegraphics[width=16 CM]{figure/024}
%%	\caption{factorial.o文件类型}
%%	\label{fig:factorial.o文件类型}
%%\end{figure}
	%\section{链\quad 接}
%\subsection{链接过程概述}
%链接(linking)是处理可重定位文件,把它们的各种符号 引用和符号定义转换为可执行文件中的合适信息的过程。而重定位是将符 号引用与符号定义进行链接的过程。\\
%
%由汇编程序生成的目标文件并不能立即就可以被执行的,其中还有许多没有 解决的问题。例如,某个源文件中的函数可能引用了另一个源文件中定义的某个变量符号,或者可能调用了某个库文件中的函数,等等。链接器的主要工作就是将有关的目标文件彼此连接,将在一个文件中引 用的符号同该符号在另外一个文件中的定义连接起来。 \\
%
%\subsection{静态链接与动态链接}
%链接处理可分为两种:静态链接和动态链接。
%\begin{itemize}
%	\item \textbf{静态链接}:静态链接过程主要是把可重定位文件依 次读入,分析各个文件的文件头,进而依次读 入各个文件的节区,并计算各个节区的虚拟内 存位置。之后,利用计算出的存储位置,对一些需要重定位的符号进行处理, 设定它们的虚拟内存地址等,最终产生一个可执行文件或者是动态链接库。 
%	
%	\item \textbf{动态链接}:动态链接所作的只是在最终的可执行程序中,记录下少量 的信息。在此可执行文件被执行时,动态链接库的全部内容将被映射虚地址空间中。动态链接程序将根据可执行程序中之前记录下来的信息找到相 应的函数指针的地址,进而能够调用这些函数,完成执行过程。 
%\end{itemize}
%
%\subsection{例程输出分析}
%
%\subsubsection{链接器调用命令}
%调用链接器生成test可执行文件的命令是:
%
%\begin{verbatim}
%      gcc test.o -o test
%\end{verbatim}
%
%\subsubsection{例程输出结果}
%
%test.o经过汇编之后得到可执行文件test,链接过程如下所示:
%
%%\begin{figure}[H]
%%	\centering
%%	\includegraphics[width=16 CM]{figure/025}
%%	\caption{链接factorial.o生成factorial}
%%	\label{链接factorial.o生成factorial}
%%\end{figure}
%
%使用如下指令可以运行test可执行文件,检验编译结果:
%\begin{verbatim}
%      ./test
%\end{verbatim}
%
%
%\begin{figure}[H]
%	\centering
%	\includegraphics[width=16 CM]{figure/0252}
%	\caption{运行test可执行文件}
%	\label{fig:运行test可执行文件}
%\end{figure}

	%\section{LLVM IR编程}
%\subsection{LLVM IR编程概述}
%%将源代码编译成可执行文件的过程需要四个步骤,并且还会 产生中间文件。读写文件都是 I/O 操作,而I/O将大大减慢GCC编译器完成编译的速度。\\
%%
%%pipe优化方式,会将上一步编译的结果通过管道传递给下一步。这将使得中间文件的读写全部在内存中完成,而不需要I/O操作,这将使得编译器的编译效率大幅度提升。\\
%%
%%可见,与-O优化选项不同,pipe优化选项并不是只针对编译过程中的某个阶段,而是一种整体性的编译优化。因此,在前文中没有对pipe优化进行介绍,在此用单独一节简要展示pipe优化的结果。
%LLVM IR(Intermediate Representation)是由代码生成器自顶向下遍历逐步翻译语法树形成的,
%你可以将任意语言的源代码编译成 LLVM IR,然后由 LLVM 后端对 LLVM IR 进行优化并编译为相
%应平台的二进制程序。LLVM IR 具有类型化、可扩展性和强表现力的特点。LLVM IR 是相对于 CPU指令集高级、但作为低级的代码中间表示的一种语言。从上述介绍中可以看出 LLVM 后端支持相当多
%的平台,我们无须担心操作系统等平台的问题,而且我们只需将代码编译成 LLVM IR,就可以由优化
%水平较高的 LLVM 后端来进行优化。此外,LLVM IR 本身更贴近汇编语言,指令集相对底层,能灵
%活地进行低级操作。\\
%
%%\begin{figure}[H]
%%	\centering
%%	\includegraphics[width=16 CM]{figure/0282}
%%	\caption{LLVM IR设计架构}
%%	\label{LLVM IR设计架构}
%%\end{figure}
%\subsection{例程输出分析}
%
%\subsubsection{LLVM 生成 LLVM IR命令}
%gcc命令下,生成test.ll文件的命令是:
%
%\begin{verbatim}
%     clang -S -emit-llvm test.c
%\end{verbatim}
%
%\subsubsection{例程输出结果}
%
%生成的可执行文件名为test.ll:
%
%%\begin{figure}[H]
%%	\centering
%%	\includegraphics[width=16 CM]{figure/0302}
%%	\caption{test.ll文件及注释-1}
%%	\label{test.ll文件及注释-1}
%%\end{figure}
%%\begin{figure}[H]
%%	\centering
%%	\includegraphics[width=16 CM]{figure/0312}
%%	\caption{test.ll文件及注释-2}
%%	\label{test.ll文件及注释-2}
%%\end{figure}
%%
%%检验a.out编译结果:
%%\begin{verbatim}
%%      ./a.out
%%\end{verbatim}
%%
%%
%%\begin{figure}[H]
%%	\centering
%%	\includegraphics[width=16 CM]{figure/028}
%%	\caption{运行a.out可执行文件}
%%	\label{fig:运行a.out可执行文件}
%%\end{figure}
%%
%%使用如下指令可以查看GCC编译器编译过程的耗时:
%%\begin{verbatim}
%%      g++ -ftime-report factorial.cpp (未使用pipe优化)
%%      g++ -pipe -ftime-report factorial.cpp (使用pipe优化)
%%\end{verbatim}
%%
%%\begin{figure}[H]
%%	\centering
%%	\includegraphics[width=16 CM]{figure/029}
%%	\caption{使用pipe优化与未使用pipe优化编译效率对比}
%%	\label{fig:使用pipe优化与未使用pipe优化编译效率对比}
%%\end{figure}
%%
%%同时也可以看出,是否使用pipe优化,只影响编译过程的耗时,并不会对最终的编译结果有任何影响,因为pipe优化只是改变了编译过程中中间文件的读写和存储方式,并不会给编译结果带来实质性的影响。
%%\begin{figure}[H]
%%	\centering
%%	\includegraphics[width=16 CM]{figure/030}
%%	\caption{使用pipe优化与未使用pipe优化编译结果可执行文件内容对比}
%%	\label{fig:使用pipe优化与未使用pipe优化编译结果可执行文件内容对比}
%%\end{figure}
	%============= 参考文献 =====================
	%\addcontentsline{toc}{section}{参考文献}
	%\bibliography{bibfile}
	\clearpage
	%=============  致谢  ======================
	%\section* {参考文献}
%\addcontentsline{toc}{section}{参考文献}
%\noindent [1]\quad GCC 官方文档 Overview 
%
%——https://gcc.gnu.org/onlinedocs/gcc-7.2.0/cpp/Overview.html\\
%
%\noindent [2]\quad GCC 官方文档 The preprocessing language——https://gcc.gnu.org/
%
%onlinedocs/cpp/The-preprocessing-language.html\#The-preprocessing-language\\
%
%\noindent [3]\quad GCC 官方文档 Preprocessor Output 
%
%——https://gcc.gnu.org/onlinedocs/cpp/Preprocessor-Output.html\#Preprocessor-Output\\
%
%\noindent [4]\quad GCC 官方文档 Conditional Syntax 
%
%——https://gcc.gnu.org/onlinedocs/cpp/Conditional-Syntax.html\#Conditional-Syntax\\
%
%\noindent [5]\quad 王春红.浅谈编译程序的工作过程[J].河东学刊,1999(06):36-37.\\
%
%\noindent [6]\quad 朱志平.高级语言中编译程序编译过程浅析[J].渭南师范学院学报,2001(02):52-54.\\
%
%%\noindent [7]\quad ubuntu环境下使用G++编译C++  
%
%——https://blog.csdn.net/qq\_28598203/article/details/52911007\\
%
%\noindent [8]\quad GCC编译器详解 
%
%%——https://blog.csdn.net/calmjiao/article/details/54987126\\
%%
%%\noindent [9]\quad gcc的编译流程详解  
%%
%%——https://www.2cto.com/kf/201610/559123.html\\
%%
%%\noindent [10]\quad gcc/g++ 实战之编译的四个过程 
%%
%%——https://www.cnblogs.com/zjiaxing/p/5557549.html\\
%%
%%\noindent [11]\quad C语言再学习 -- GCC编译过程  
%%
%%——https://blog.csdn.net/qq\_29350001/article/details/53339861\\
%%
%%\noindent [12]\quad C语言中的预编译宏定义 
%%
%%——https://blog.csdn.net/a\_ran/article/details/42711707\\
%%
%%\noindent [13]\quad GCC编译器优化选项分析及具体优化了什么  
%%
%%——https://blog.csdn.net/gatieme/article/details/48898261\\
%%
%%%\noindent [14]\quad AT\&T格式汇编学习  
%%
%%——https://blog.csdn.net/subfate/article/details/50676467\\
%%
%%%\noindent [15]\quad AT\&T汇编格式  
%%
%%——https://blog.csdn.net/u011414997/article/details/43560453\\
%%
%%\noindent [16]\quad 链接原理  
%%
%%——http://www.cnblogs.com/xiaomanon/p/4210016.html\\
%%
%%\noindent [17]\quad GCC编译过程与动态链接库和静态链接库  
%%
%%——https://www.cnblogs.com/king-lps/p/7757919.html\\
%%
%%\noindent [18]\quad 静态链接库与动态链接库----C/C++  
%%
%%——https://blog.csdn.net/freestyle4568world/article/details/49817799\\
%
%

	%\newpage
%\appendix
%
%%%附录第一个章节
%%\section{附录1}
%
%
%%%变量列举
%
%%\begin{table}[H]
%%\caption{gcc在编译cpp文件时或g++在编译c文件和cpp文件时额外加入的宏}
%%\centering
%%\begin{tabular}{l}
%%\toprule
%%\textbf{\quad \quad \quad \quad \quad \quad 宏名称} \\
%%\midrule
%%\#define \_\_GXX\_WEAK\_\_ 1\\
%%\#define \_\_cplusplus 1\\
%%\#define \_\_DEPRECATED 1\\
%%\#define \_\_GNUG\_\_ 4\\
%%\#define \_\_EXCEPTIONS 1\\
%%\#define \_\_private\_extern\_\_ extern\\
%%\bottomrule
%%\end{tabular}
%%\end{table}
%
%
%\section{附录1}
%\textcolor[rgb]{0.98,0.00,0.00}{\quad \quad \quad \quad \quad \quad \quad \quad \quad \quad \quad \quad  \textbf{test.c源代码}}
%
%\begin{python}
%#include<stdio.h>
%int main()
%{
%	int i,n,f;
%	scanf("%d",&n);
%	i=2;
%	f=1;
%	while(i<=n)
%	{
%		f=f*i;
%		i=i+1;
%	}
%	printf("%d\n",f);
%}
%\end{python}
%
%\begin{lstlisting}[language=c++]
%#include<iostream>
%using namespace std;
%int main()
%{
%int i, n, f;
%cin >> n;
%i = 2;
%f = 1;
%while (i <= n)
%{
%%f = f * i;
%i = i + 1;
%}
%cout << f << endl;
%}
%\end{lstlisting}

	
\end{document}
%%%%%%%%%% 结束 %%%%%%%%%%