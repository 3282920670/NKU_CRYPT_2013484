%\section{LLVM IR编程}
%\subsection{LLVM IR编程概述}
%%将源代码编译成可执行文件的过程需要四个步骤,并且还会 产生中间文件。读写文件都是 I/O 操作,而I/O将大大减慢GCC编译器完成编译的速度。\\
%%
%%pipe优化方式,会将上一步编译的结果通过管道传递给下一步。这将使得中间文件的读写全部在内存中完成,而不需要I/O操作,这将使得编译器的编译效率大幅度提升。\\
%%
%%可见,与-O优化选项不同,pipe优化选项并不是只针对编译过程中的某个阶段,而是一种整体性的编译优化。因此,在前文中没有对pipe优化进行介绍,在此用单独一节简要展示pipe优化的结果。
%LLVM IR(Intermediate Representation)是由代码生成器自顶向下遍历逐步翻译语法树形成的,
%你可以将任意语言的源代码编译成 LLVM IR,然后由 LLVM 后端对 LLVM IR 进行优化并编译为相
%应平台的二进制程序。LLVM IR 具有类型化、可扩展性和强表现力的特点。LLVM IR 是相对于 CPU指令集高级、但作为低级的代码中间表示的一种语言。从上述介绍中可以看出 LLVM 后端支持相当多
%的平台,我们无须担心操作系统等平台的问题,而且我们只需将代码编译成 LLVM IR,就可以由优化
%水平较高的 LLVM 后端来进行优化。此外,LLVM IR 本身更贴近汇编语言,指令集相对底层,能灵
%活地进行低级操作。\\
%
%%\begin{figure}[H]
%%	\centering
%%	\includegraphics[width=16 CM]{figure/0282}
%%	\caption{LLVM IR设计架构}
%%	\label{LLVM IR设计架构}
%%\end{figure}
%\subsection{例程输出分析}
%
%\subsubsection{LLVM 生成 LLVM IR命令}
%gcc命令下,生成test.ll文件的命令是:
%
%\begin{verbatim}
%     clang -S -emit-llvm test.c
%\end{verbatim}
%
%\subsubsection{例程输出结果}
%
%生成的可执行文件名为test.ll:
%
%%\begin{figure}[H]
%%	\centering
%%	\includegraphics[width=16 CM]{figure/0302}
%%	\caption{test.ll文件及注释-1}
%%	\label{test.ll文件及注释-1}
%%\end{figure}
%%\begin{figure}[H]
%%	\centering
%%	\includegraphics[width=16 CM]{figure/0312}
%%	\caption{test.ll文件及注释-2}
%%	\label{test.ll文件及注释-2}
%%\end{figure}
%%
%%检验a.out编译结果:
%%\begin{verbatim}
%%      ./a.out
%%\end{verbatim}
%%
%%
%%\begin{figure}[H]
%%	\centering
%%	\includegraphics[width=16 CM]{figure/028}
%%	\caption{运行a.out可执行文件}
%%	\label{fig:运行a.out可执行文件}
%%\end{figure}
%%
%%使用如下指令可以查看GCC编译器编译过程的耗时:
%%\begin{verbatim}
%%      g++ -ftime-report factorial.cpp (未使用pipe优化)
%%      g++ -pipe -ftime-report factorial.cpp (使用pipe优化)
%%\end{verbatim}
%%
%%\begin{figure}[H]
%%	\centering
%%	\includegraphics[width=16 CM]{figure/029}
%%	\caption{使用pipe优化与未使用pipe优化编译效率对比}
%%	\label{fig:使用pipe优化与未使用pipe优化编译效率对比}
%%\end{figure}
%%
%%同时也可以看出,是否使用pipe优化,只影响编译过程的耗时,并不会对最终的编译结果有任何影响,因为pipe优化只是改变了编译过程中中间文件的读写和存储方式,并不会给编译结果带来实质性的影响。
%%\begin{figure}[H]
%%	\centering
%%	\includegraphics[width=16 CM]{figure/030}
%%	\caption{使用pipe优化与未使用pipe优化编译结果可执行文件内容对比}
%%	\label{fig:使用pipe优化与未使用pipe优化编译结果可执行文件内容对比}
%%\end{figure}