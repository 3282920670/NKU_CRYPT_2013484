%\section* {参考文献}
%\addcontentsline{toc}{section}{参考文献}
%\noindent [1]\quad GCC 官方文档 Overview 
%
%——https://gcc.gnu.org/onlinedocs/gcc-7.2.0/cpp/Overview.html\\
%
%\noindent [2]\quad GCC 官方文档 The preprocessing language——https://gcc.gnu.org/
%
%onlinedocs/cpp/The-preprocessing-language.html\#The-preprocessing-language\\
%
%\noindent [3]\quad GCC 官方文档 Preprocessor Output 
%
%——https://gcc.gnu.org/onlinedocs/cpp/Preprocessor-Output.html\#Preprocessor-Output\\
%
%\noindent [4]\quad GCC 官方文档 Conditional Syntax 
%
%——https://gcc.gnu.org/onlinedocs/cpp/Conditional-Syntax.html\#Conditional-Syntax\\
%
%\noindent [5]\quad 王春红.浅谈编译程序的工作过程[J].河东学刊,1999(06):36-37.\\
%
%\noindent [6]\quad 朱志平.高级语言中编译程序编译过程浅析[J].渭南师范学院学报,2001(02):52-54.\\
%
%%\noindent [7]\quad ubuntu环境下使用G++编译C++  
%
%——https://blog.csdn.net/qq\_28598203/article/details/52911007\\
%
%\noindent [8]\quad GCC编译器详解 
%
%%——https://blog.csdn.net/calmjiao/article/details/54987126\\
%%
%%\noindent [9]\quad gcc的编译流程详解  
%%
%%——https://www.2cto.com/kf/201610/559123.html\\
%%
%%\noindent [10]\quad gcc/g++ 实战之编译的四个过程 
%%
%%——https://www.cnblogs.com/zjiaxing/p/5557549.html\\
%%
%%\noindent [11]\quad C语言再学习 -- GCC编译过程  
%%
%%——https://blog.csdn.net/qq\_29350001/article/details/53339861\\
%%
%%\noindent [12]\quad C语言中的预编译宏定义 
%%
%%——https://blog.csdn.net/a\_ran/article/details/42711707\\
%%
%%\noindent [13]\quad GCC编译器优化选项分析及具体优化了什么  
%%
%%——https://blog.csdn.net/gatieme/article/details/48898261\\
%%
%%%\noindent [14]\quad AT\&T格式汇编学习  
%%
%%——https://blog.csdn.net/subfate/article/details/50676467\\
%%
%%%\noindent [15]\quad AT\&T汇编格式  
%%
%%——https://blog.csdn.net/u011414997/article/details/43560453\\
%%
%%\noindent [16]\quad 链接原理  
%%
%%——http://www.cnblogs.com/xiaomanon/p/4210016.html\\
%%
%%\noindent [17]\quad GCC编译过程与动态链接库和静态链接库  
%%
%%——https://www.cnblogs.com/king-lps/p/7757919.html\\
%%
%%\noindent [18]\quad 静态链接库与动态链接库----C/C++  
%%
%%——https://blog.csdn.net/freestyle4568world/article/details/49817799\\
%
%
